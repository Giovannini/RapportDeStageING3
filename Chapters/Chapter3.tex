% Chapter 3

\chapter{Les apports techniques} % Main chapter title

\label{apports} % For referencing the chapter elsewhere, use \ref{apports_techniques}

\lhead{Troisième partie. \emph{Les apports techniques}} % This is for the header on each page - perhaps a shortened title

%----------------------------------------------------------------------------------------
%----------------------------------------------------------------------------------------

Le site Cadremploi.fr est découpé en différentes parties.
En effet, il dispose d'une partie publique, qui est celle dans laquelle un utilisateur peut rechercher une offre en fonction de critères et y postuler.
Une seconde partie, destinée aux recruteurs, leur permet de saisir une offre d'emploi qui sera ensuite consultable depuis la partie client.
J'ai travaillé durant ce stage sur la partie concernant les recruteurs, ou "Espace Recruteur".
L'espace recruteur est aujourd'hui trop ancien, difficilement maintenable et utilisable.
Une refonte totale en était nécessaire.
Mon stage consistait ainsi à participer avec l'équipe Cadremploi à la création d'un nouvel Espace Recruteur, plus moderne.


\section{Description générale de la nouvelle application}
\label{sec:Description generale de l'application}
L'application Espace Recruteur permet à des recruteurs d'entreprise de publier dans la partie publique du site Cadremploi une offre destinée à des cadres et cadres supérieurs.
Néanmoins, il s'agit de plus qu'un simple formulaire.
En effet, en plus du remplissage d'un formulaire esthétique permettant de rédiger l'offre qu'il souhaite déposer, l'espace recruteur propose plusieurs services.
Le recruteur ayant un compte sur l'espace recruteur du site Cadremploi dispose d'un panneau de contrôle lui permettant de gérer ses offres et d'en consulter les informations associées.
De plus, une équipe de Figaro CLASSIFIEDS supervise via une seconde application dédiée, la publication des offres.

%---------------------------------------------------------------
\subsection{Backoffice}
\label{sub:Backoffice}
Le backoffice est une application interne à Cadremploi permettant de gérer toutes les offres existantes.
Chaque offre créée par un recruteur se doit d'être complètement sous contrôle de manière à pouvoir gérer tout cas inattendu.
Ainsi, cette application est développée en parallèle à l'espace recruteur par notre équipe, comme un panneau de gestion de toutes offres rédigées par les recruteurs.
Un backoffice existe déjà pour l'espace recruteur courant, mais il est aussi agé que l'espace recruteur actuel et les nouveautés apportées dans cette nouvelle version demandent une telle adaptation que le développement d'un nouveau backoffice est nécessaire.
\begin{figure}[b]
  \begin{center}
    \hspace*{-1in}
    \includegraphics[width=1.3\textwidth]{Pictures/backoffice.png}
    \label{pic:suivi client backoffice}
    \caption{Le suivi de l'évolution du client dans le backoffice}
  \end{center}
\end{figure}
Les fonctionnalités nécessaires sur ce nouveau backoffice sont:
\begin{itemize}
  \item Le login "en tant que" un autre recruteur:
  Il est possible depuis le backoffice de se connecter pour éditer une offre comme si on était le recruteur qui l'avait rédigé et ainsi y apporter tous les changements désirés.
  Cela permet un contrôle parfait sur une offre puisque toutes les actions permises pour le recruteur le sont aussi pour le superviseur.
  Il est ainsi possible de gérer les problèmes potentiels en effectuant les modifications nécessaires à la place du recruteur.
  \item La gestion de la discrimination des offres en fonction de leur contenu:
  Une des fonctionnalité de l'espace recruteur est la détection de termes discriminants dans les champs textuels d'une offre.
  Ainsi, si une offre est détectée comme présentant des termes discriminants, elle n'est pas directement publiée sur le site.
  Une offre discriminante ne sera visible que sur le backoffice d'où elle sera relue par des responsables de la relation client (RC).
  Les RC sont chargés de traiter, si nécessaire avec le client, les offres discriminantes afin de les rendre acceptables dans les plus brefs délais.
  Le but de cette procédure est de ne pas handicaper le client et de régler les problèmes rencontrés rapidement.
  \item Le suivi de l'évolution du client au cours de la création de son offre (\ref{pic:suivi client backoffice}).
  Cette fonctionnalité permet à un RC de pouvoir se rendre compte du parcours du client lorsque celui-ci rédige son offre (l'ordre dans lequel il a rempli les champs, le temps qu'il a mis à les remplir, ...) et donc de pouvoir l'aider au mieux lors de sa prise de contact.
  Les événements de modification effectués par le recruteur au cours de sa saisie sont donc visibles par le RC consultant une offre donnée.
\end{itemize}

%---------------------------------------------------------------
\subsection{Le panneau de contrôle du recruteur}
%---------------------------------------------------------------
\subsubsection{Statistiques}
\label{subs:Statistiques}
\begin{figure}[h]
  \begin{center}
    \hspace*{-1in}
    \includegraphics[width=1.2\textwidth]{Pictures/stats.png}
    \label{pic:suivi client backoffice}
    \caption{Les mesures statistiques proposées sur l'espace recruteur}
  \end{center}
\end{figure}
Un système de statistiques mesurant la rentabilité de l'annonce est mis en place pour que le recruteur puisse suivre l'évolution de son offre.
Ce système de statistiques est totalement externalisé de l'application et a été développé par une équipe différente de celle de Cadremploi.
Il permet ainsi à tout FIGARO CLASSIFIEDS de récupérer des statistiques sur l'utilisation de l'espace recruteur de Cadremploi.
L'incorporation de cette API de statistiques ainsi que la récupération des informations nécessaires étaient deux points importants dans le développement du nouvel Espace Recruteur.
%---------------------------------------------------------------
\subsubsection{Label qualité}
\label{subs:Label qualité}
Une des autres manière de garantir la qualité des offres présentes sur Cadremploi est la création d'un label qualité.
Il s'agit en somme d'un système de classement des offres qui fera passer les offres considérées comme étant de qualité devant celles ne l'étant pas dans la liste visible sur l'espace public.
Cela permet d'inciter le recruteur à proposer des annonces avec des descriptions complètes ou des vidéos présentant l'entreprise, de manière à proposer un taux d'offres attractives plus important.
Cette fonctionnalité n'est pas implémentée à ce jour.
%---------------------------------------------------------------
\subsubsection{Event sourcing}
\label{subs:Event sourcing}
Enfin, une gestion fine des modifications apportées par les recruteurs à leurs offres sera faite.
En effet, chaque modification faite par un recruteur sur son annonce sera enregistrée de manière à suivre son évolution au cours du processus de mise en ligne d'annonce.
Elle permettra par exemple aux RC traitant avec les recruteurs de pouvoir comprendre clairement la façon dont ils ont procédés et ainsi mieux les aider.
Cette gestion permet de plus, comme je l'expliquerai plus tard, de pouvoir revenir à un état précédent d'une offre, dans le cas d'un recruteur ayant fait une erreur de manipulation.

\paragraph{}
Seule la mise en place du pattern Event Sourcing avait débuté à mon arrivée et j'ai ainsi participé au développement des autres fonctionnalités.

\section{Description générale de la nouvelle application}
\label{sec:Description generale de l'application}
L'application Espace Recruteur permet à des recruteurs d'entreprise de publier dans la partie publique du site Cadremploi une offre destinée à des cadres et cadres supérieurs.
Il s'agit néanmoins de plus qu'un simple formulaire.
En effet, en plus du remplissage d'un formulaire esthétique permettant de décrire l'offre qu'il souhaite déposer, l'espace recruteur propose plusieurs services.
Le recruteur ayant un compte sur l'espace recruteur du site Cadremploi dispose d'un panneau de contrôle lui permettant de gérer ses offres et d'en consulter les informations associées.
De plus, une équipe de Figaro CLASSIFIEDS supervise via une seconde application dédiée, la publication des offres.
%----------------------------------------------------------------------------------------

\subsection{Backoffice}
\label{sub:Backoffice}
Le backoffice est une application interne à Cadremploi permettant de gérer toutes les offres existantes.
Chaque offre créé par un recruteur se doit d'être complètement sous contrôle de manière à pouvoir gérer tout cas inattendu.
Ainsi, cette application est développée en parallèle à l'espace recruteur par notre équipe, comme un panneau de gestion de toutes offres rédigées par les recruteurs.
Un backoffice existe déjà pour l'espace recruteur courant, mais les nouveautés apportés dans cette nouvelle version demandent une telle adaptation que le développement d'un nouveau backoffice est nécessaire.
Les fonctionnalités nécessaires sur ce nouveau backoffice sont:
\begin{itemize}
  \item Le login "en tant que" un autre recruteur:
  Il est possible depuis le backoffice de se connecter pour éditer une offre comme si on était le recruteur qui l'avait rédigé et ainsi y apporter tous les changements désiré.
  Cela permet un contrôle parfait sur une offre puisque toutes les actions permises pour le recruteur le sont aussi pour le superviseur.
  Il est ainsi possible de gérer les problèmes potentiels en effectuant les modifications nécessaires à la place du recruteur.
  \item La gestion de la discrimination des offres en fonction de leur contenu:
  Une des fonctionnalité de l'espace recruteur est la détection de termes discriminants dans les champs textuels d'une offre.
  Ainsi, si une offre est détectée comme présentant des termes discriminants, elle n'est pas directement publiée sur le site.
  Une offre discriminante ne sera visible que sur le backoffice d'où elle sera relue par des responsables de la relation client (RC).
  Les RC sont chargés de traiter, si nécessaire avec le client, les offres discriminantes afinde les rendre acceptables dans les plus brefs délais.
  Le but de cette procédure est de ne pas handicaper le client et de régler les problèmes rencontrés rapidement.
  \item Le suivi de l'évolution du client au cours de la création de son offre.
  Cette fonctionnalité permet à un RC de pouvoir se rendre compte du parcours du client lorsque celui-ci rédige son offre (l'ordre dans lequel il a rempli les champs, le temps qu'il a mis à les remplir, ...) et donc de pouvoir l'aider au mieux lors de sa prise de contact.
  Les événements de modification effectués par le recruteur au cours de sa saisie sont donc visibles par le RC consultant une offre donnée.
\end{itemize}

\subsection{Statistiques}
\label{sub:Statistiques}
Un système de statistiques mesurant la rentabilité de l'annonce est mis en place pour que le recruteur puisse suivre l'évolution de son offre.
Ce système de statistiques est totalement externalisé de l'application et il permet ainsi à tout Figaro Classifieds de récupérer des statistiques sur l'utilsation de l'espace recruteur de Cadremploi et il a été développé par une équipe différente de celle de Cadremploi.
L'incorporation de cette API de statistiques ainsi que la récupération des informations nécessaires était un point important dans le développement du nouvel Espace Recruteur.

\subsection{Label qualité}
\label{sub:Label qualité}
Une des autres manière de garantir la qualité des offres présente sur Cadremploi est la création d'un label qualité.
Il s'agit en somme d'un système de classement des offres qui fera passer les offres considérées comme étant de qualité devant celles ne l'étant pas dans la liste visible sur l'espace public.
Il s'agit d'un système permettant d'inciter le recruteur à proposer des annonces avec des descriptions complètes ou des vidéos présentant l'entreprise, de manière à proposer un taux d'offres attractives plus important.
Cette fonctionnalité n'est pas implémentée à ce jour.

\subsection{Event sourcing}
\label{sub:Event sourcing}
Enfin, une gestion fine des modifications apportés par les recruteurs à leurs offres sera faite.
En effet, chaque modification faite par un recruteur sur son annonce sera enregistré de manière à suivre son évolution au cours du processus de mise en ligne d'annonce.
Elle permettra par exemple aux RC traitant avec les recruteurs de pouvoir comprendre clairement la façon dont ils ont procédés et ainsi mieux les aider.

\paragraph{}
Seule la mise en place du pattern Event Sourcing avait débuté à mon arrivée et j'ai ainsi participé au développement des autres fonctionnalités.

\section{Aperçu des technologies et des techniques utilisées}

\subsection{Langages}
\label{sub:Langages}
\subsubsection{Backend}
\label{subs:Backend}
Le nouvel espace recruteur est géré en backend via le framework Play! dans sa dernière version (2.4.2) et sous le langage Scala.
Il s'agit d'un couple qui permet d'écrire du code rapidement et de façon maintenable.
Play est l'un des framework les plus connus dans le monde Java et offre de nombreux avantages:
\begin{itemize}
  \item Il supporte le rechargement à chaud.
  Il suffit de lancer Play via sa console en mode développement et il prendra automatiquement en compte à chaud les changements effectués sur le code, mais aussi les templates ou le routage.
  Cela contribue grandement au gain de productivité qu'offre Play.
  \item En plus de s’appuyer sur du code à typage statique, Play propose la sécurité du typage à d’autres endroits et notamment sur les templates ou sur le routage des différents contrôleurs.
  Un certain nombre de problèmes sont alors mis en lumière directement à l’étape de la compilation.
  \item Il permet d'exécuter des tests, notamment sur la couche web à plusieurs niveaux:
  On peut par exemple tester un contrôleur en démarrant un serveur web (donc via HTTP) ou, sans le démarrer, en appelant simplement le contrôleur avec le contexte qui va bien, le tout de manière simple et rapide à l’exécution.
  \item Il est sans état et basé sur des entrées sorties non bloquantes et permet ainsi une capacité à monter en charge très intéressante
  \item Il supporte nativement REST, JSON, Websocket entre autres et se présente donc comme un framework moderne
\end{itemize}
Un des points négatifs de ce choix en backend est l'utilisation de fait obligatoire de SBT qui nous a posé de nombreux problèmes
\subsubsection{Frontend}
\label{subs:Frontend}
Du côté front, l'équipe a utilisé le framework AngularJS ainsi que la librairie D3.js qui permettent des animations dynamiques.
\subsection{Environnements et outils}
Le développement de l'espace recruteur s'est fait sous le framework Play! et IntelliJ Idea était l'IDE utilisé par l'équipe pour développer, puisqu'il offre une bonne intégration de ce framework.
J'avais déjà souvent utilisé cet IDE pour les différents projets que j'ai eu à rendre tout au long de ma scolarité à l'EISTI, ainsi son utilisation ne m'a posé aucun problème.
L'équipe étant nombreuse et l'application étant sujete à grossir avec le temps, un outil de gestion de versions a été utilisée.
Git est l'outil le plus répandu et le plus efficace connu, c'est ce que l'équipe Cadremploi a utilisé.
Il s'agissait aussi d'un outil que j'avais beaucoup utilisé mais de manière très simpliste et j'ai appris à m'en servir d'une toute autre façon, bien plus complète, pendant ce stage.
L'équipe Cadremploi utilise aussi Jenkins comme outil d'intégration continue pour déployer les nouveautés sur les différents environnement de production utilisés.
Cet outil avait été brièvement présenté lors du cours d'Outils de Développement mais jamais utilisé.
Bien que nous avions été tenté de mettre en place un Jenkins avec mon équipe de Projet de Fin d'Étude, cela ne s'est jamais fait par manque de temps et puisqu'un besoin concret ne s'est jamais montré.
Je n'ai pas eu de mal à l'utiliser puisqu'une fois configuré, cet outil est réelement simple d'utilisation.
Plusieurs environnements de développement sont utilisés par Cadremploi comme pour les autres équipes de Figaro Classifieds:
\begin{itemize}
  \item{Environnement de développement}, où les développeurs postent régulièrement les améliorations qu'ils mettent en place. Jenkins n'est pas utilisé pour cet environnement puisque les changements y sont publiés directement via Git.
  \item{Environnement de recette}, où la Project Owner vérifie que les changements apportés fonctionnent effectivement et les valide.
  \item{Environnement de pré-production} qui est un environnement qui ressemble autant que possible à l'environnement de production et où sont surtout testés les architectures
  \item{Environnement de production} qui est la dernière étape, contenant l'application utlisée réelement par les utilisateurs
\end{itemize}
J'ai été habitué durant ma scolarité à l'EISTI à utiliser la majorité des outils présentés précédemment puisque j'étais uniquement étranger à Jenkins. J'ai malgré tout grandement progressé dans mon utilisation de ces outils, particulièrement pour Git puisque j'ai redécouvert l'outil puisque le projet auquel je participais differais grandement en taille de ceux auquels j'avais contribué auparavant.
%Potentiellement, Rundeck, Kibana, Puppet.

\section{Une équipe nombreuse}
\label{sec:Une équipe nombreuse}
Cadremploi, a mon arrivée, comptait alors 10 membres dont 9 développeurs.
Il s'agit d'une équipe d'une taille importante pour un projet Agile.
Ainsi, de l'organisation ainsi qu'une bonne coordination étaient nécessaires.

\subsection{Revue de code}
\label{sub:Revue de code}
Le développement d'applications demande très souvent à une équipe de personnes de travailler ensemble.
Plus l'équipe grandit plus il est compliqué de garantir une coordination correcte et une bonne organisation.
Les équipes d'ingénieur sont spécialement soumises à ce genre de problèmes puisque du code est quotidiennement partagé entre plusieurs personnes au sein d'une entreprise.
La revue de code aide à partager le savoir et les bonnes pratiques.
\paragraph{}
Le flux de production de l'équipe Cadremploi demande à chaque membre de l'équipe à passer un ticket terminé dans un état "R7 Equipe".
Le ticket doit alors être revu par un autre développeur de l'équipe.
Ce dernier doit analyser les changements fait au code, s'assurer qu'ils n'introduisent pas de troubles dans l'application et vérifier que le code produit suit bien les normes de code imposées par l'équipe.
\paragraph{}
La revue de code par d'autres membres de l'équipe offre de nombreux avantages autres qu'une simple vérification du code.
En effet, il est clair que n'avoir qu'un unique membre de l'équipe responsable d'un point critique sur l'application est dangereux.
La revue de code distribue le savoir au sein de l'équipe et permet d'informer plusieurs membres d'un changement.
De plus, elle stimule les conversations portant sur la structure du code, l'architecture de l'application et permet ainsi à de nouveaux venus, comme je l'étais, de s'adapter rapidement et donc de devenir productif plus rapidement.
\paragraph{}
Il ne m'est pas possible de réelement comparer ce mode de fonctionnement avec mes projets précédents.
En effet, mes projets passés se faisaient dans des équipes plus réduites, et le temps manquait pour la revue de code.
Cela ne nous empêchait pas de suivre la règle du boyscoot, qui prescrit de toujours laisser une fonction, un module dans un meilleur état que celui dans lequel on l'a trouvé, et qui évite la "possession" du code par un membre de l'équipe.
Cette règle est plus ou moins comprise dans la revue de code de Cadremploi qui offre plus d'avantages encore.

\subsection{Utilisation de Git}
\label{sub:utilisation de Git}
Git est un outil de contrôle de version flexible et puissant.
Il permet notamment de mettre en commun du code de manière à ce que plusieurs personnes puissent contribuer à un même projet.
Git propose énormément d'options et il est parfois compliqué de déterminer la meilleure façon de contribuer à un projet sans nuire aux autres participants, la mise en commun de code étant parfois, si ce n'est souvent, source de conflits entre deux versions existantes écrites par différentes personnes.
Notre équipe est constituée de 9 développeurs, et nous ne pouvions échapper à de tels scénarios.
C'est pourquoi quelques semaines après mon arrivée, nous avons convenu avec l'équipe de bonnes pratiques.
Leurs buts étaient multiples, il s'agissait ici de limiter les problèmes de mise en commun de code souvent trop nombreux et bloquants, de simplifier la relecture par les autres membres de l'équipe lors des revues de code mais aussi de clarifier la lecture de l'historique d'évolution de l'application.

\subsubsection{Concertations d'équipe}
\label{subs:Concertations d'équipe}
L'utilisation de Git a ainsi été sujet à une concertation de toute l'équipe, mais cela a été le cas pour de nombreux autres points.
J'avais mentionné précédemment que la communication était un point très important dans une méthodologie Agile, et c'est pourquoi des points ponctuels étaient .

%statuts d'une offre, utilisation du fonctionnel, bonnes pratiques, ...

\section{CQRS: Command Query Responsibility Seggregation}
\label{sec:CQRS: Command Query Responsibility Seggregation}
CQRS (Command and Query Responsibility Segregation)est un modèle d'acrhitecture plutôt récent dont le principe repose, comme son nom l'indique, sur la séparation entre l'écriture et la lecture de l'information.
Le but de ce fonctionnement est notamment la production d'un système extensible, distribuable.
Nous avons suivi lors du développement de l'espace recruteur ce pattern puisque la séparations des composants de traitement (les "commands") et de restitution (les "queries") de l'information offrait une architecture très intéressante de laquelle nous avons tiré de nombreux bénéfices tels que la suppression du risque d'effets de bord ou l'allègement des classes de service.

%-------------------------------------------------------------------------
\subsection{Organisation d'une application basée sur une telle architecture}
Par opposition à une architecture du type 3 tiers, dont les services permettant d'accéder aux données se confondent avec ceux qui vont agir sur ces même données, l'architecture CQRS sépare volontairement les composantsrequêtant les données de ceux qui les modifient.
Une telle séparation facilite l'organisation de l'application puisque des composants différents sont utilisés pour des actions différentes.
De plus, elle permet de répartir plus facilement les charges dans le cas d'une infrastructure distribuée.

%-------------------------------------------------------------------------
\subsection{Explication détaillée}
\label{sub:explication}
%-------------------------------------------------------------------------
\subsubsection{La couche de modification des données: Command}
\label{subs:La couche de modification des données: Command}
Cette couche concentre toutes les modifications des données, qu'il s'agisse de création, de suppression ou de mise à jour.
Une commande représente une action destinée à être exécutée, une intention, et n'est pas une simple demande d'altération de donnée.
En effet, on retrouve dans la commande la raison de la modification de la donnée.
Dans le cas de l'espace recruteur du site Cadremploi.fr, ces commandes ont pour but d'enregistrer les altérations effectuées par les événements côté client, et on retrouve des commandes nommées 'ModifierDescriptionPosteCommand' ou 'PayerOffreCommand' par exemple.
De tels noms sont ainsi très expressifs et permettent de clarifier la cause de la modification des données.
%-------------------------------------------------------------------------
\subsubsection{La couche de lecture des données: Query}
\label{subs:La couche de lecture des données: Query}
La partie Query de ce modèle se base sur le fait que les objets du modèle sont volumineux et qu'il est possible de s'en passer.
Cette couche fonctionne ainsi uniquement en lecture seule, aucune modification n'est apportée aux données.
En effet le besoin représenté ici est celui d'une lecture dans un cas d'utilisation bien précis, l'objectif est d'aller extrêmement vite.
Le pattern Query consiste alors à exécuter une requête précise en base et de restituer un objet DTO concis qui pourra être utilisé directement.
Cela signifie que les contrôles sont réduits au minimum, mais cette méthode permet de récupérer seulement les données dont on a besoin en une fois, en se passant ainsi de parcourir plusieurs tables à travers desquelles les données seraient éparpillées.
Le modèle CQRS de base peut s'avérer problématique puisque la grosse majorité des sollicitations du système se fait sur cette partie.
%-------------------------------------------------------------------------
\subsubsection{Le domaine}
\label{subs:Le domaine}
Le domaine est la zone où est concentré toute la connaissance métier de l'application.
C'est de là que chaque commande est analysée et qu'il est décidé si l'on donne suite à chacune d'entre elle.
On y trouve ainsi les différents objets permettant de pratiquer les contrôles nécessaires entre autre.

% Pour le site Cadremploi, cette requête est une requête ElasticSearch et est donc extrêmement rapide.
% La partie de Read de l'application se résume ainsi à la simple exécution d'une requête ES et au remplissage de DTO.
%-------------------------------------------------------------------------
\subsection{Implémentation}
\label{sub:Implémentation}
\subsubsection{Infrastructure}
\label{subs:Infrastructure}
L'utilisation de l'architecture CQRS est rendue possible grâce à plusieurs concepts tirés du Domain Driven Design (DDD).
\paragraph{Les aggrégats}
Il s'agit des objets sur lesquels agissent les commandes.
Ce concept, que l'on trouve dans le Domain Driven Design, correspond à un groupe d'objets que l'on peut traiter comme une unité.
Par exemple, l'aggregat 'Entreprise', que l'on retrouve dans l'espace recruteur de cadremploi.fr est constitué des objets 'NomEntreprise', 'AdresseFacturation' ou encore 'Logo'.
\paragraph{Les repositories (ou dépôts)}
Les différentes instances de cet aggrégat sont stockées dans un objet venu aussi du DDD, un repository (dépôt en français).
Ces objets servent d'intermédiaires entre \ref(subs:Le domaine){le domaine} et la base de donnée, qui est pour l'espace recruteur une base PostgreSQL.
On retrouve dans ces objets des méthodes du type 'create' permettant de créer une nouvelle instance d'un aggrégat ou 'save' qui permet de sauvegarder les changements relatifs à un aggrégat.
\paragraph{Récupération des données}
L'utilisation des repository via PostgreSQL est intéressante en écriture, mais ne permet pas une lecture intense et rapide.
L'équipe Cadremploi utilise Elasticsearch de manière à pouvoir indexer ces données et ainsi pouvoir chercher et trier rapidement les données dont elle a besoin.
C'est l'utilisation de cet outil qui permet d'exécuter des requêtes de manière extrêmement rapide.
%-------------------------------------------------------------------------
\subsubsection{Les commandes}
\label{subs:Les commandes}
L'utilisation du pattern Command est rendu possible grâce à plusieurs objets.
Un exécuteur de commande ("CommandExecutor") expose une méthode permettant d'exécuter une commande donnée.
Cet exécuteur se charge de
\begin{itemize}
  \item retrouver l'aggrégat visé par la commande s'il existe
  \item vérifier que l'auteur de l'action est autorisé à l'effectuer
  \item récupérer le CommandHandler associé à la commande, c'est à dire l'objet responsable de traiter la commande reçue
\end{itemize}

\section{Event sourcing}
% http://martinfowler.com/eaaDev/EventSourcing.html
% https://github.com/eventstore/eventstore/wiki/Event-Sourcing-Basics
Une des demandes pour le nouvel espace recruteur est de pouvoir suivre avec précision les actions de l'utilisateur.
Ainsi chaque modification qu'il apportera à une de ses offres ou à son profil par exemple doit être enregistrée.
%----------------------------------------------------------------------------------------
\textit{"L'idée fondamentale derrière l'Event Sourcing est d'assurer que chaque modification de l'état d'une application est capturée dans un objet événement et que ces événements sont eux-mêmes stockés dans l'ordre dans lequel ils ont été appliqués à l'application."} (Martin Fowler)
%----------------------------------------------------------------------------------------
\subsection{Fonctionnement désiré}
\label{sub:Fonctionnement desire}
Il est nécessaire de pouvoir suivre l'évolution de l'annonce que créé un utilisateur.
Concrètement, à chaque fois que l'utilisateur modifie son annonce, on veut générer un événement qui informe notre système de cette modification.
Toutes ces données pourront ainsi être utilisées pour faire des statistiques ainsi que des contrôles.
Une des utilités de ce suivi est par exemple de pouvoir contrôler les éditions abusives d'annonces déjà en ligne.
Il est ainsi possible de suivre de très près la façon dont l'utilisateur procède, ou les champs qui posent problème.

%----------------------------------------------------------------------------------------
\subsection{Fonctionnement technique}
%----------------------------------------------------------------------------------------
\subsubsection{Enregistrement de l'événement}
\label{subs:Enregistrement de l'evenement}
Concrètement, derrière chaque champ formulaire que l'utilisateur remplit, un watcher d'Angular est présent.
Ainsi lorsque ce watcher indique que le champ a été modifié, on récupère la valeur qu'il contient et envoyons cette demande de modification vers le backend.
Cette "demande de modification" est traitée comme une commande et génère alors un ou plusieurs événements qui résument la modification effectuée et sont stockés dans une base de donnée, conformément au modèle de Write du pattern CQRS décrit ci-dessus.
Chacune des modifications faite par l'utilisateur est ainsi enregistrée, et il est possible de connaître l'état du système à tout instant t donné en interrogeant la base de données.
%----------------------------------------------------------------------------------------
\subsubsection{Utilisation de l'événement}
\label{subs:Utilisation de l'evenement}
Si l'enregistrement en base se passe bien, l'événement est ensuite publié sur un EventBus, un objet Akka qui permet entre autre d'envoyer un message à des groupes d'acteurs, pour être écouté par des objets de type Projection.
Ces projections écoutent seulement une partie des événements selon leurs responsabilité et modifient en conséquence un objet DTO qu'elles indexent.
Ce traitement est fait de manière asynchrone, via le système d'acteurs Akka.
Il permet donc d'optimiser le modèle CQRS vu précédemment: la commande est exécutée rapidement puisque l'événement est sauvegardé de suite et le reste du traitement est effectué de manière asynchrone.
Cela permet d'informer quasi instantannément l'utilisateur que son action a été prise en compte.
D'autre part, en acceptant de rendre asynchrone et donc pas forcément instantannée la mise à jour de la partie Read du modèle, on améliore grandement la performance ressentie par l'utilisateur.
Le système est donc tenu à jour en lecture et ainsi, à l'appel d'une query, les projections peuvent être interrogées pour récupérer la donnée de manière très rapide.


\section{Domain Driven Development}


%----------------------------------------------------------------------------------------
%----------------------------------------------------------------------------------------
\section{Organisation}
comparaison entre les plannings prévisionnels et réels

%----------------------------------------------------------------------------------------

%----------------------------------------------------------------------------------------
%----------------------------------------------------------------------------------------

\section{Auto-évaluation}
respect des délais, autonomie, qualité du travail, apports à l'équipe

%----------------------------------------------------------------------------------------
%----------------------------------------------------------------------------------------

\section{Résultats et prolongements possibles}
le site tourne et continuera de fonctionner encore

%----------------------------------------------------------------------------------------
