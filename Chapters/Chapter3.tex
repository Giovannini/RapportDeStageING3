% Chapter 3

\chapter{Être développeur à Cadremploi} % Main chapter title

\label{apports} % For referencing the chapter elsewhere, use \ref{apports_techniques}

\lhead{Troisième partie. \emph{Être développeur à Cadremploi}} % This is for the header on each page - perhaps a shortened title

%-------------------------------------------------------------------------------
\begin{figure}[h]
  \begin{center}
    \hspace*{-1in}
    \includegraphics[width=0.5\textwidth]{Pictures/CE_logo.png}
  \end{center}
\end{figure}
%-------------------------------------------------------------------------------
\paragraph{}
Le site Cadremploi.fr est découpé en différentes parties qui sont en réalité des applications à part.
La partie publique est celle dans laquelle un utilisateur peut rechercher une offre en fonction de critères et y postuler.
Une seconde partie, destinée aux recruteurs, leur permet de saisir une offre d'emploi qui sera ensuite consultable depuis la partie publique.
J'ai travaillé durant ce stage sur la partie concernant les recruteurs, ou "Espace Recruteur".
Il était en effet à mon arrivée difficilement maintenable, utilisable et son interface était obsolète.
Une refonte totale en était nécessaire.
%-------------------------------------------------------------------------------
\paragraph{}
Mon stage consistait ainsi à participer avec l'équipe Cadremploi à la création d'un nouvel Espace Recruteur, plus moderne.
Je vais détailler dans cette partie le fonctionnement du nouvel espace recruteur ainsi que la façon dont l'équipe s'est organisé pour le développer.


\section{Description générale de la nouvelle application}
\label{sec:Description generale de l'application}
L'application Espace Recruteur permet à des recruteurs d'entreprise de publier dans la partie publique du site Cadremploi une offre destinée à des cadres et cadres supérieurs.
Néanmoins, il s'agit de plus qu'un simple formulaire.
En effet, en plus du remplissage d'un formulaire esthétique permettant de rédiger l'offre qu'il souhaite déposer, l'espace recruteur propose plusieurs services.
Le recruteur ayant un compte sur l'espace recruteur du site Cadremploi dispose d'un panneau de contrôle lui permettant de gérer ses offres et d'en consulter les informations associées.
De plus, une équipe de Figaro CLASSIFIEDS supervise via une seconde application dédiée, la publication des offres.

%---------------------------------------------------------------
\subsection{Backoffice}
\label{sub:Backoffice}
Le backoffice est une application interne à Cadremploi permettant de gérer toutes les offres existantes.
Chaque offre créée par un recruteur se doit d'être complètement sous contrôle de manière à pouvoir gérer tout cas inattendu.
Ainsi, cette application est développée en parallèle à l'espace recruteur par notre équipe, comme un panneau de gestion de toutes offres rédigées par les recruteurs.
Un backoffice existe déjà pour l'espace recruteur courant, mais il est aussi agé que l'espace recruteur actuel et les nouveautés apportées dans cette nouvelle version demandent une telle adaptation que le développement d'un nouveau backoffice est nécessaire.
\begin{figure}[b]
  \begin{center}
    \hspace*{-1in}
    \includegraphics[width=1.3\textwidth]{Pictures/backoffice.png}
    \label{pic:suivi client backoffice}
    \caption{Le suivi de l'évolution du client dans le backoffice}
  \end{center}
\end{figure}
Les fonctionnalités nécessaires sur ce nouveau backoffice sont:
\begin{itemize}
  \item Le login "en tant que" un autre recruteur:
  Il est possible depuis le backoffice de se connecter pour éditer une offre comme si on était le recruteur qui l'avait rédigé et ainsi y apporter tous les changements désirés.
  Cela permet un contrôle parfait sur une offre puisque toutes les actions permises pour le recruteur le sont aussi pour le superviseur.
  Il est ainsi possible de gérer les problèmes potentiels en effectuant les modifications nécessaires à la place du recruteur.
  \item La gestion de la discrimination des offres en fonction de leur contenu:
  Une des fonctionnalité de l'espace recruteur est la détection de termes discriminants dans les champs textuels d'une offre.
  Ainsi, si une offre est détectée comme présentant des termes discriminants, elle n'est pas directement publiée sur le site.
  Une offre discriminante ne sera visible que sur le backoffice d'où elle sera relue par des responsables de la relation client (RC).
  Les RC sont chargés de traiter, si nécessaire avec le client, les offres discriminantes afin de les rendre acceptables dans les plus brefs délais.
  Le but de cette procédure est de ne pas handicaper le client et de régler les problèmes rencontrés rapidement.
  \item Le suivi de l'évolution du client au cours de la création de son offre (\ref{pic:suivi client backoffice}).
  Cette fonctionnalité permet à un RC de pouvoir se rendre compte du parcours du client lorsque celui-ci rédige son offre (l'ordre dans lequel il a rempli les champs, le temps qu'il a mis à les remplir, ...) et donc de pouvoir l'aider au mieux lors de sa prise de contact.
  Les événements de modification effectués par le recruteur au cours de sa saisie sont donc visibles par le RC consultant une offre donnée.
\end{itemize}

%---------------------------------------------------------------
\subsection{Le panneau de contrôle du recruteur}
%---------------------------------------------------------------
\subsubsection{Statistiques}
\label{subs:Statistiques}
\begin{figure}[h]
  \begin{center}
    \hspace*{-1in}
    \includegraphics[width=1.2\textwidth]{Pictures/stats.png}
    \label{pic:suivi client backoffice}
    \caption{Les mesures statistiques proposées sur l'espace recruteur}
  \end{center}
\end{figure}
Un système de statistiques mesurant la rentabilité de l'annonce est mis en place pour que le recruteur puisse suivre l'évolution de son offre.
Ce système de statistiques est totalement externalisé de l'application et a été développé par une équipe différente de celle de Cadremploi.
Il permet ainsi à tout FIGARO CLASSIFIEDS de récupérer des statistiques sur l'utilisation de l'espace recruteur de Cadremploi.
L'incorporation de cette API de statistiques ainsi que la récupération des informations nécessaires étaient deux points importants dans le développement du nouvel Espace Recruteur.
%---------------------------------------------------------------
\subsubsection{Label qualité}
\label{subs:Label qualité}
Une des autres manière de garantir la qualité des offres présentes sur Cadremploi est la création d'un label qualité.
Il s'agit en somme d'un système de classement des offres qui fera passer les offres considérées comme étant de qualité devant celles ne l'étant pas dans la liste visible sur l'espace public.
Cela permet d'inciter le recruteur à proposer des annonces avec des descriptions complètes ou des vidéos présentant l'entreprise, de manière à proposer un taux d'offres attractives plus important.
Cette fonctionnalité n'est pas implémentée à ce jour.
%---------------------------------------------------------------
\subsubsection{Event sourcing}
\label{subs:Event sourcing}
Enfin, une gestion fine des modifications apportées par les recruteurs à leurs offres sera faite.
En effet, chaque modification faite par un recruteur sur son annonce sera enregistrée de manière à suivre son évolution au cours du processus de mise en ligne d'annonce.
Elle permettra par exemple aux RC traitant avec les recruteurs de pouvoir comprendre clairement la façon dont ils ont procédés et ainsi mieux les aider.
Cette gestion permet de plus, comme je l'expliquerai plus tard, de pouvoir revenir à un état précédent d'une offre, dans le cas d'un recruteur ayant fait une erreur de manipulation.

\paragraph{}
Seule la mise en place du pattern Event Sourcing avait débuté à mon arrivée et j'ai ainsi participé au développement des autres fonctionnalités.

\section{Description générale de la nouvelle application}
\label{sec:Description generale de l'application}
L'application Espace Recruteur permet à des recruteurs d'entreprise de publier dans la partie publique du site Cadremploi une offre destinée à des cadres et cadres supérieurs.
Il s'agit néanmoins de plus qu'un simple formulaire.
En effet, en plus du remplissage d'un formulaire esthétique permettant de décrire l'offre qu'il souhaite déposer, l'espace recruteur propose plusieurs services.
Le recruteur ayant un compte sur l'espace recruteur du site Cadremploi dispose d'un panneau de contrôle lui permettant de gérer ses offres et d'en consulter les informations associées.
De plus, une équipe de Figaro CLASSIFIEDS supervise via une seconde application dédiée, la publication des offres.
%----------------------------------------------------------------------------------------

\subsection{Backoffice}
\label{sub:Backoffice}
Le backoffice est une application interne à Cadremploi permettant de gérer toutes les offres existantes.
Chaque offre créé par un recruteur se doit d'être complètement sous contrôle de manière à pouvoir gérer tout cas inattendu.
Ainsi, cette application est développée en parallèle à l'espace recruteur par notre équipe, comme un panneau de gestion de toutes offres rédigées par les recruteurs.
Un backoffice existe déjà pour l'espace recruteur courant, mais les nouveautés apportés dans cette nouvelle version demandent une telle adaptation que le développement d'un nouveau backoffice est nécessaire.
Les fonctionnalités nécessaires sur ce nouveau backoffice sont:
\begin{itemize}
  \item Le login "en tant que" un autre recruteur:
  Il est possible depuis le backoffice de se connecter pour éditer une offre comme si on était le recruteur qui l'avait rédigé et ainsi y apporter tous les changements désiré.
  Cela permet un contrôle parfait sur une offre puisque toutes les actions permises pour le recruteur le sont aussi pour le superviseur.
  Il est ainsi possible de gérer les problèmes potentiels en effectuant les modifications nécessaires à la place du recruteur.
  \item La gestion de la discrimination des offres en fonction de leur contenu:
  Une des fonctionnalité de l'espace recruteur est la détection de termes discriminants dans les champs textuels d'une offre.
  Ainsi, si une offre est détectée comme présentant des termes discriminants, elle n'est pas directement publiée sur le site.
  Une offre discriminante ne sera visible que sur le backoffice d'où elle sera relue par des responsables de la relation client (RC).
  Les RC sont chargés de traiter, si nécessaire avec le client, les offres discriminantes afinde les rendre acceptables dans les plus brefs délais.
  Le but de cette procédure est de ne pas handicaper le client et de régler les problèmes rencontrés rapidement.
  \item Le suivi de l'évolution du client au cours de la création de son offre.
  Cette fonctionnalité permet à un RC de pouvoir se rendre compte du parcours du client lorsque celui-ci rédige son offre (l'ordre dans lequel il a rempli les champs, le temps qu'il a mis à les remplir, ...) et donc de pouvoir l'aider au mieux lors de sa prise de contact.
  Les événements de modification effectués par le recruteur au cours de sa saisie sont donc visibles par le RC consultant une offre donnée.
\end{itemize}

\subsection{Statistiques}
\label{sub:Statistiques}
Un système de statistiques mesurant la rentabilité de l'annonce est mis en place pour que le recruteur puisse suivre l'évolution de son offre.
Ce système de statistiques est totalement externalisé de l'application et il permet ainsi à tout Figaro Classifieds de récupérer des statistiques sur l'utilsation de l'espace recruteur de Cadremploi et il a été développé par une équipe différente de celle de Cadremploi.
L'incorporation de cette API de statistiques ainsi que la récupération des informations nécessaires était un point important dans le développement du nouvel Espace Recruteur.

\subsection{Label qualité}
\label{sub:Label qualité}
Une des autres manière de garantir la qualité des offres présente sur Cadremploi est la création d'un label qualité.
Il s'agit en somme d'un système de classement des offres qui fera passer les offres considérées comme étant de qualité devant celles ne l'étant pas dans la liste visible sur l'espace public.
Il s'agit d'un système permettant d'inciter le recruteur à proposer des annonces avec des descriptions complètes ou des vidéos présentant l'entreprise, de manière à proposer un taux d'offres attractives plus important.
Cette fonctionnalité n'est pas implémentée à ce jour.

\subsection{Event sourcing}
\label{sub:Event sourcing}
Enfin, une gestion fine des modifications apportés par les recruteurs à leurs offres sera faite.
En effet, chaque modification faite par un recruteur sur son annonce sera enregistré de manière à suivre son évolution au cours du processus de mise en ligne d'annonce.
Elle permettra par exemple aux RC traitant avec les recruteurs de pouvoir comprendre clairement la façon dont ils ont procédés et ainsi mieux les aider.

\paragraph{}
Seule la mise en place du pattern Event Sourcing avait débuté à mon arrivée et j'ai ainsi participé au développement des autres fonctionnalités.

%----------------------------------------------------------------------------------------
\section{Une équipe nombreuse}
\label{sec:Une équipe nombreuse}
%----------------------------------------------------------------------------------------
Cadremploi, a mon arrivée, comptait alors 10 membres dont 9 développeurs.
Il s'agit d'une équipe d'une taille importante pour un projet Agile.
Ainsi, de l'organisation ainsi qu'une bonne coordination étaient nécessaires.

%----------------------------------------------------------------------------------------
\subsection{Revue de code}
\label{sub:Revue de code}
Le développement d'applications demande très souvent à une équipe de personnes de travailler ensemble.
Plus l'équipe grandit plus il est compliqué de garantir une coordination correcte et une bonne organisation.
Les équipes d'ingénieur sont spécialement soumises à ce genre de problèmes puisque du code est quotidiennement partagé entre plusieurs personnes au sein d'une entreprise.
La revue de code aide à partager le savoir et les bonnes pratiques.
%----------------------------------------------------------------------------------------
\paragraph{}
Le flux de production de l'équipe Cadremploi demande à chaque membre de l'équipe à passer un ticket terminé dans un état "R7 Equipe".
Le ticket doit alors être revu par un autre développeur de l'équipe.
Ce dernier doit analyser les changements fait au code, s'assurer qu'ils n'introduisent pas de récession dans l'application et vérifier que le code produit suit bien les normes de code imposées par l'équipe.
%----------------------------------------------------------------------------------------
\paragraph{}
La revue de code par d'autres membres de l'équipe offre de nombreux avantages autres qu'une simple vérification du code.
En effet, il est clair que n'avoir qu'un unique membre de l'équipe responsable d'un point critique sur l'application est dangereux.
La revue de code distribue le savoir au sein de l'équipe et permet d'informer plusieurs membres d'un changement.
De plus, elle stimule les conversations portant sur la structure du code, l'architecture de l'application et permet ainsi à de nouveaux venus, comme je l'étais, de s'adapter rapidement et donc de devenir productif plus rapidement.
%----------------------------------------------------------------------------------------
\paragraph{}
Il ne m'est pas possible de réelement comparer le mode de fonctionnement de la revue de code de l'équipe Cadremploi.
En effet, mes projets passés se faisaient dans des équipes plus réduites, et du temps était rarament pris pour la revue de code.
Cela ne nous empêchait pas de suivre des règles évitant la "possession" du code par un membre de l'équipe, mais rien de plus.
%----------------------------------------------------------------------------------------
\paragraph{}
La revue de code effectuée avec rigueur se montre extrêmement utile, il s'agit d'une des activités qui m'a permi d'apprendre le plus au sein de l'équipe Cadremploi.
En effet, en favorisant la communication mais aussi les critiques, positives, au sein de l'équipe, elle permet une amélioration continue des méthodes de l'équipe.

%----------------------------------------------------------------------------------------
\subsection{Concertation d'équipe}
\label{sub:Concertation d'équipe}
%----------------------------------------------------------------------------------------
\subsubsection{L'exemple de l'utilisation de Git}
\label{subs:Utilisation de Git}
Git est un outil de contrôle de version flexible et puissant.
Il permet notamment de mettre en commun du code de manière à ce que plusieurs personnes puissent contribuer à un même projet.
Les options qu'il propose sont nombreuses et il est parfois compliqué de déterminer la meilleure façon de contribuer à un projet sans nuire aux autres contributeurs, la mise en commun de code étant parfois, si ce n'est souvent, source de conflits entre deux versions écrites par différentes personnes.
Notre équipe est constituée de 9 développeurs, et nous ne pouvions échapper à de tels scénarios.
C'est pourquoi quelques semaines après mon arrivée, nous avons convenu avec l'équipe de bonnes pratiques.
Leurs buts étaient multiples, il s'agissait ici de limiter les problèmes de mise en commun de code souvent trop nombreux et bloquants, de simplifier la relecture par les autres membres de l'équipe lors des revues de code mais aussi de clarifier la lecture de l'historique d'évolution de l'application.
%----------------------------------------------------------------------------------------
\subsubsection{Les réunions}
\label{subs:Les réunions}
%----------------------------------------------------------------------------------------
\paragraph{}
L'utilisation de Git a ainsi été sujet à une concertation de toute l'équipe, mais cela a été le cas pour de nombreux autres points.
J'avais mentionné précédemment que la communication était un point très important dans une méthodologie Agile, et c'est pourquoi des points ponctuels étaient nécessaires.
Il est souvent arrivé que notre équipe se réunisse dans le but de discuter autour de bonnes pratiques au niveau du code, de l'utilisation du fonctionnel qui est nouveau pour la plupart des membres de l'équipe issus d'un univers Java, mais aussi sur des sujets d'architecture et de modélisation.
Ces réunions étaient souvent organisées surtout dans le but de présenter une ligne directrice, mais tout participant était écouté et l'idée principale se trouvait alors renforcée par les discussions qui émergeaient.

\section{Aperçu des technologies et des techniques utilisées}

\subsection{Langages}
\label{sub:Langages}
\subsubsection{Backend}
\label{subs:Backend}
\begin{figure}[h]
  \begin{center}
    \includegraphics[width=0.3\textwidth]{Pictures/play_logo.png}
    \hspace{1in}
    \includegraphics[width=0.15\textwidth]{Pictures/scala_logo.png}
  \end{center}
\end{figure}
\paragraph{Le framework Play}
Le nouvel espace recruteur est géré en backend via le framework Play! dans sa dernière version (2.4.2) et sous le langage Scala.
Il s'agit d'un couple qui permet d'écrire du code rapidement et de façon maintenable.
Play est l'un des framework les plus connus dans le monde Java et offre de nombreux avantages:
\begin{itemize}
  \item Il supporte le rechargement à chaud.
  Il suffit de lancer Play via sa console en mode développement et il prendra automatiquement en compte à chaud les changements effectués sur le code, mais aussi les templates ou le routage.
  Cela contribue grandement au gain de productivité qu'offre Play.
  \item En plus de s’appuyer sur du code à typage statique, Play propose la sécurité du typage à d’autres endroits et notamment sur les templates ou sur le routage des différents contrôleurs.
  Un certain nombre de problèmes sont alors mis en lumière directement à l’étape de la compilation.
  \item Il permet d'exécuter des tests, notamment sur la couche web à plusieurs niveaux:
  On peut par exemple tester un contrôleur en démarrant un serveur web (donc via HTTP) ou, sans le démarrer, en appelant simplement le contrôleur avec le bon contexte, le tout de manière simple et rapide à l’exécution.
  \item Il est sans état et basé sur des entrées sorties non bloquantes et permet ainsi une capacité à monter en charge très intéressante
  \item Il supporte nativement REST, JSON, Websocket entre autres et se présente donc comme un framework moderne
\end{itemize}
Un des points négatifs de ce choix en backend est l'utilisation de fait obligatoire de SBT qui nous a posé de nombreux problèmes.
En effet, cet outil était souvent lent, compliqué d'utilisation et ralentissait souvent le développement notamment via ses résolutions de dépendances interminables.
%-------------------------------------------------------------------------
\paragraph{Scala}
Scala a été le langage choisi pour le développement de l'espace recruteur pour sa flexibilité et sa richesse.
Il s'agit d'un langage permettant d'implémenter des fonctionnalités complexes, notamment liées à l'asynchronisme, tout en maintenant un code clair et de qualité.
Ce langage a aussi permi d'utiliser la puissance d'Akka ainsi que de nombreuses librairies propres à son écosystème et à celui de Java.
\paragraph{Akka}
Cadremploi utilise Akka pour diffuser des événements de manière asynchrone au sein des applications Espace Recruteur et Backoffice.
Akka est aussi utilisé pour programmer l'exécution de scripts récurrents.

%--------------------------------------------------------------
\subsubsection{Frontend}
\label{subs:Frontend}
\begin{figure}[h]
  \begin{center}
    \includegraphics[width=0.4\textwidth]{Pictures/angular_logo.png}
  \end{center}
\end{figure}
Du côté front, l'équipe a utilisé le framework AngularJS ainsi que la librairie D3.js qui permettent des animations dynamiques.
Ce choix de technologie peut être soutenu pour les points suivants:
\begin{itemize}
  \item Angular permet une maintenabilité de l'application aisée, notamment en demandant une structure de type MVC au développeur, ainsi qu'en utilisant du HTMl, qui est déclaratif, pour définir l'interface utilisateur.
  \item Il s'agit de plus d'un framework flexible puisqu'il est possible de composer une application en prototypant des composants qui sont de plus facilement testables.
\end{itemize}


\paragraph{}
Cette stack reactive assure à l'équipe Cadremploi une consistence ainsi qu'une clareté architecturale.
Elle permet la création d'une application responsive et fiable, donnant aux utilisateurs et aux clients une confiance résidant dans son architecture et son implémentation.

\subsection{Environnements et outils}
\paragraph{IntelliJ}
\label{par:IntelliJ}
\begin{wrapfigure}{l}{0.15\textwidth}
  \vspace{-2.5em}
  \begin{center}
    \includegraphics[width=0.15\textwidth]{Pictures/intellij_logo.png}
  \end{center}
\end{wrapfigure}
Le développement de l'espace recruteur s'est fait sous le framework Play! et IntelliJ Idea était l'IDE utilisé par l'équipe pour développer, puisqu'il offre une bonne intégration de ce framework.
J'avais déjà souvent utilisé cet IDE pour différents projets que j'ai eu à rendre tout au long de ma scolarité à l'EISTI, ainsi son utilisation ne m'a posé aucun problème.
\paragraph{Git}
\label{par:Git}
L'équipe étant nombreuse et l'application étant sujete à grossir avec le temps, un outil de gestion de versions a été utilisé.
Git est l'outil le plus répandu et le plus efficace connu, c'est ce que l'équipe Cadremploi a utilisé.
Il s'agissait aussi d'un outil que j'avais beaucoup utilisé mais de manière très simpliste et j'ai appris à m'en servir d'une toute autre façon, bien plus complète, pendant ce stage.
\paragraph{Jenkins}
\label{par:Jenkins}
\begin{wrapfigure}{rH}{0.25\textwidth}
  \begin{center}
    \includegraphics[width=0.15\textwidth]{Pictures/jenkins_logo.png}
  \end{center}
\end{wrapfigure}
L'équipe Cadremploi utilise aussi Jenkins comme outil d'intégration continue pour déployer les nouveautés sur les différents environnements de production utilisés (recette, intégration), les mises en production et préproductions n'étant pas gérées directement par l'équipe.
Cet outil avait été brièvement présenté lors du cours d'Outils de Développement mais jamais utilisé.
Jenkins est un outil open source permettant de compiler et de tester un projet de manière continue.
Il aide de fait les développeurs à intégrer facilement des changements à une application.
Bien que nous avions été tenté de mettre en place un Jenkins avec mon équipe de Projet de Fin d'Étude, cela ne s'est jamais fait par manque de temps et puisqu'un besoin concret ne s'est jamais présenté.
Je n'ai pas eu de mal à l'utiliser puisqu'une fois configuré, cet outil est réelement simple d'utilisation.
Plusieurs environnements de développement sont utilisés par Cadremploi comme pour les autres équipes de Figaro Classifieds:
\begin{itemize}
  \item{Environnement de développement}, où les développeurs postent régulièrement les améliorations qu'ils mettent en place. Jenkins n'est pas utilisé pour cet environnement puisque les changements y sont publiés directement via Git.
  \item{Environnement de recette}, où la Project Owner vérifie que les changements apportés fonctionnent effectivement et les valide.
  \item{Environnement de pré-production} qui est un environnement qui ressemble autant que possible à l'environnement de production où sont effectués les derniers tests avant la mise en production
  \item{Environnement de production} qui est la dernière étape, contenant l'application utlisée réelement par les utilisateurs
  \item{Environnement d'intégration}, il s'agit d'un environnement supplémentaire qui est préféré pour les tests fait sur l'infrastructure du projet.
\end{itemize}
J'ai été habitué durant ma scolarité à l'EISTI à utiliser la majorité des outils présentés précédemment puisque j'étais uniquement étranger à Jenkins.
J'ai malgré tout grandement progressé dans leur utilisation et particulièrement pour Git que j'ai redécouvert; le projet auquel je participais differais grandement en taille de ceux auquels j'avais contribué auparavant.
%Potentiellement, Rundeck, Kibana, Puppet.

\section{Architecture de l'application}
\label{sec:Architecture de l'application}
%-------------------------------------------------------------------------------
L'espace recruteur est dispose d'une architecture interne assez singulière et je n'avais jamais eu à faire à ce genre d'application auparavant.
En effet, l'équipe Cadremploi a mis en place un modèle d'architecture de type CQRS couplé à une gestion d'état basé sur de l'Event Sourcing.
%-------------------------------------------------------------------------------
\paragraph{CQRS: Command Query Responsibility Seggregation}
\label{par:CQRS: Command Query Responsibility Seggregation}
Dans les systèmes de gestion de donnée traditionnels, les commandes, c'est à dire la mise à jour des données, et les requêtes son exécutées sur un seul regroupement d'entitées regroupées dans une unique base de donnée.
CQRS est un modèle d'architecture plutôt récent dont le principe repose, comme son nom l'indique, sur la séparation entre l'écriture et la lecture de l'information.
Nous avons suivi lors du développement de l'espace recruteur ce pattern puisque la séparations des composants de traitement (les "commands") et de restitution (les "queries") de l'information offrait une architecture très intéressante de laquelle nous avons tiré de nombreux bénéfices tels que la suppression du risque d'effets de bord ou l'allègement des classes de service.
%-------------------------------------------------------------------------------
\paragraph{Event Sourcing}
\label{par:Event Sourcing}
La gestion commune de l'état d'un système consiste à enregistrer l'état courant des objets le composant et de reporter chaque changement effectué sur le système en modifiant directement son état, notamment via une mise à jour en base de donnée.
L'idée fondamentale de l'Event Sourcing est d'assurer que chaque changement appliqué à l'état d'une application peut être capturée dans un objet de type événement.
Ce n'est plus l'état du système qui est enregistré mais les événements qui ont mené le système à l'état dans lequel il est actuellement.
Cette façon de penser permet notamment d'obtenir un log complet de tous les changements effectués, et donc une traçabilité ainsi qu'une aisance de debug importante.
Il est ainsi possible de pouvoir reconstruire tous les états passés de l'application.
%-------------------------------------------------------------------------------
\paragraph{}
Je vais présenter dans cette partie ce type singulier d'architecture, montrer la façon dont il permet de mettre en place un système extensible et distribuable et enfin expliquer la façon dont elle a été mise en place sur la nouvelle version de l'espace recruteur de cadremploi.fr.

%-------------------------------------------------------------------------------
\input{"Chapters/Chapter3/architecture/organisation"}
\input{"Chapters/Chapter3/architecture/implementation"}
%-------------------------------------------------------------------------------

