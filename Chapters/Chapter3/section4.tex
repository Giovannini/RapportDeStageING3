\section{Une équipe nombreuse}
\label{sec:Une équipe nombreuse}
Cadremploi, a mon arrivée, comptait alors 10 membres dont 9 développeurs.
Il s'agit d'une équipe d'une taille importante pour un projet Agile.
Ainsi, de l'organisation ainsi qu'une bonne coordination étaient nécessaires.

\subsection{Revue de code}
\label{sub:Revue de code}
Le développement d'applications demande très souvent à une équipe de personnes de travailler ensemble.
Plus l'équipe grandit plus il est compliqué de garantir une coordination correcte et une bonne organisation.
Les équipes d'ingénieur sont spécialement soumises à ce genre de problèmes puisque du code est quotidiennement partagé entre plusieurs personnes au sein d'une entreprise.
La revue de code aide à partager le savoir et les bonnes pratiques.
\paragraph{}
Le flux de production de l'équipe Cadremploi demande à chaque membre de l'équipe à passer un ticket terminé dans un état "R7 Equipe".
Le ticket doit alors être revu par un autre développeur de l'équipe.
Ce dernier doit analyser les changements fait au code, s'assurer qu'ils n'introduisent pas de troubles dans l'application et vérifier que le code produit suit bien les normes de code imposées par l'équipe.
\paragraph{}
La revue de code par d'autres membres de l'équipe offre de nombreux avantages autres qu'une simple vérification du code.
En effet, il est clair que n'avoir qu'un unique membre de l'équipe responsable d'un point critique sur l'application est dangereux.
La revue de code distribue le savoir au sein de l'équipe et permet d'informer plusieurs membres d'un changement.
De plus, elle stimule les conversations portant sur la structure du code, l'architecture de l'application et permet ainsi à de nouveaux venus, comme je l'étais, de s'adapter rapidement et donc de devenir productif plus rapidement.
\paragraph{}
Il ne m'est pas possible de réelement comparer ce mode de fonctionnement avec mes projets précédents.
En effet, mes projets passés se faisaient dans des équipes plus réduites, et le temps manquait pour la revue de code.
Cela ne nous empêchait pas de suivre la règle du boyscoot, qui prescrit de toujours laisser une fonction, un module dans un meilleur état que celui dans lequel on l'a trouvé, et qui évite la "possession" du code par un membre de l'équipe.
Cette règle est plus ou moins comprise dans la revue de code de Cadremploi qui offre plus d'avantages encore.

\subsection{Utilisation de Git}
\label{sub:utilisation de Git}
Git est un outil de contrôle de version flexible et puissant.
Il propose énormément d'options de workflow et il est parfois compliqué
Git is a flexible and powerful version control system. While Git offers significant functionality over legacy centralized tools like CVS and Subversion, it also presents so many options for workflow that it can be difficult to determine what is the best method to commit code to a project. The following are the guidelines I like to use for most software projects contained within a Git repository. They aren't applicable to every Git project (especially those hosted on drupal.org or GitHub), but I've found that they help ensure that our own projects end up with a reasonable repository history.
