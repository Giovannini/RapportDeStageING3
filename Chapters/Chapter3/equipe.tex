%----------------------------------------------------------------------------------------
\section{Une équipe nombreuse}
\label{sec:Une équipe nombreuse}
%----------------------------------------------------------------------------------------
Cadremploi, a mon arrivée, comptait alors 10 membres dont 9 développeurs.
Il s'agit d'une équipe d'une taille importante pour un projet Agile.
Ainsi, de l'organisation ainsi qu'une bonne coordination étaient nécessaires.

%----------------------------------------------------------------------------------------
\subsection{Revue de code}
\label{sub:Revue de code}
Le développement d'applications demande très souvent à une équipe de personnes de travailler ensemble.
Plus l'équipe grandit plus il est compliqué de garantir une coordination correcte et une bonne organisation.
Les équipes d'ingénieur sont spécialement soumises à ce genre de problèmes puisque du code est quotidiennement partagé entre plusieurs personnes au sein d'une entreprise.
La revue de code aide à partager le savoir et les bonnes pratiques.
%----------------------------------------------------------------------------------------
\paragraph{}
Le flux de production de l'équipe Cadremploi demande à chaque membre de l'équipe à passer un ticket terminé dans un état "R7 Equipe".
Le ticket doit alors être revu par un autre développeur de l'équipe.
Ce dernier doit analyser les changements fait au code, s'assurer qu'ils n'introduisent pas de récession dans l'application et vérifier que le code produit suit bien les normes de code imposées par l'équipe.
%----------------------------------------------------------------------------------------
\paragraph{}
La revue de code par d'autres membres de l'équipe offre de nombreux avantages autres qu'une simple vérification du code.
En effet, il est clair que n'avoir qu'un unique membre de l'équipe responsable d'un point critique sur l'application est dangereux.
La revue de code distribue le savoir au sein de l'équipe et permet d'informer plusieurs membres d'un changement.
De plus, elle stimule les conversations portant sur la structure du code, l'architecture de l'application et permet ainsi à de nouveaux venus, comme je l'étais, de s'adapter rapidement et donc de devenir productif plus rapidement.
%----------------------------------------------------------------------------------------
\paragraph{}
Il ne m'est pas possible de réelement comparer le mode de fonctionnement de la revue de code de l'équipe Cadremploi.
En effet, mes projets passés se faisaient dans des équipes plus réduites, et du temps était rarament pris pour la revue de code.
Cela ne nous empêchait pas de suivre des règles évitant la "possession" du code par un membre de l'équipe, mais rien de plus.
%----------------------------------------------------------------------------------------
\paragraph{}
La revue de code effectuée avec rigueur se montre extrêmement utile, il s'agit d'une des activités qui m'a permi d'apprendre le plus au sein de l'équipe Cadremploi.
En effet, en favorisant la communication mais aussi les critiques, positives, au sein de l'équipe, elle permet une amélioration continue des méthodes de l'équipe.

%----------------------------------------------------------------------------------------
\subsection{Concertation d'équipe}
\label{sub:Concertation d'équipe}
%----------------------------------------------------------------------------------------
\subsubsection{L'exemple de l'utilisation de Git}
\label{subs:Utilisation de Git}
Git est un outil de contrôle de version flexible et puissant.
Il permet notamment de mettre en commun du code de manière à ce que plusieurs personnes puissent contribuer à un même projet.
Les options qu'il propose sont nombreuses et il est parfois compliqué de déterminer la meilleure façon de contribuer à un projet sans nuire aux autres contributeurs, la mise en commun de code étant parfois, si ce n'est souvent, source de conflits entre deux versions écrites par différentes personnes.
Notre équipe est constituée de 9 développeurs, et nous ne pouvions échapper à de tels scénarios.
C'est pourquoi quelques semaines après mon arrivée, nous avons convenu avec l'équipe de bonnes pratiques.
Leurs buts étaient multiples, il s'agissait ici de limiter les problèmes de mise en commun de code souvent trop nombreux et bloquants, de simplifier la relecture par les autres membres de l'équipe lors des revues de code mais aussi de clarifier la lecture de l'historique d'évolution de l'application.
%----------------------------------------------------------------------------------------
\subsubsection{Les réunions}
\label{subs:Les réunions}
%----------------------------------------------------------------------------------------
\paragraph{}
L'utilisation de Git a ainsi été sujet à une concertation de toute l'équipe, mais cela a été le cas pour de nombreux autres points.
J'avais mentionné précédemment que la communication était un point très important dans une méthodologie Agile, et c'est pourquoi des points ponctuels étaient nécessaires.
Il est souvent arrivé que notre équipe se réunisse dans le but de discuter autour de bonnes pratiques au niveau du code, de l'utilisation du fonctionnel qui est nouveau pour la plupart des membres de l'équipe issus d'un univers Java, mais aussi sur des sujets d'architecture et de modélisation.
Ces réunions étaient souvent organisées surtout dans le but de présenter une ligne directrice, mais tout participant était écouté et l'idée principale se trouvait alors renforcée par les discussions qui émergeaient.
