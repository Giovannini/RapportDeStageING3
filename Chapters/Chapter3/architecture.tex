\section{Architecture de l'application}
\label{sec:Architecture de l'application}
L'espace recruteur dispose d'une architecture interne assez singulière et je n'avais jamais eu à faire à ce genre d'application auparavant.
En effet, l'équipe Cadremploi a mis en place un modèle d'architecture de type CQRS couplé à une gestion d'état basé sur de l'Event Sourcing.

\paragraph{CQRS: Command Query Responsibility Seggregation}
\label{par:CQRS: Command Query Responsibility Seggregation}
CQRS (Command and Query Responsibility Segregation) est un modèle d'architecture plutôt récent dont le principe repose, comme son nom l'indique, sur la séparation entre l'écriture et la lecture de l'information.
Dans les systèmes de gestion de donnée traditionnels, les commandes, c'est à dire la mise à jour des données, et les requêtes son exécutées sur un seul regroupement d'entitées regroupées dans une unique base de donnée.
Nous avons suivi lors du développement de l'espace recruteur ce pattern puisque la séparations des composants de traitement (les "commands") et de restitution (les "queries") de l'information offrait une architecture très intéressante de laquelle nous avons tiré de nombreux bénéfices tels que la suppression du risque d'effets de bord ou l'allègement des classes de service.

\paragraph{Event Sourcing}
\label{par:Event Sourcing}
L'idée fondamentale de l'Event Sourcing est d'assurer que chaque changement appliqué à l'état d'une application peut être capturée dans un objet événement et que ces objets, eux-même stockés dans le même ordre que celui dans lequel ils ont été appliqués, permettent de connapitre l'état de cette application à tout instant t.
Il est ainsi possible, en plus de seulement requêter ces événements, de pouvoir reconstruire les états passés de l'application.

\paragraph{}
Je vais présenter dans cette partie ce type singulier d'architecture, montrer la façon dont il permet de mettre en place un système extensible et distribuable et enfin expliquer la façon dont elle a été mise en place sur la nouvelle version de l'espace recruteur de cadremploi.fr.

\input{"Chapters/Chapter3/architecture/organisation"}
\input{"Chapters/Chapter3/architecture/implementation"}
