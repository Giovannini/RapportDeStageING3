% Pour le site Cadremploi, cette requête est une requête ElasticSearch et est donc extrêmement rapide.
% La partie de Read de l'application se résume ainsi à la simple exécution d'une requête ES et au remplissage de DTO.
%-------------------------------------------------------------------------
\subsection{Implémentation}
\label{sub:Implémentation}
\subsubsection{Infrastructure}
\label{subs:Infrastructure}
L'utilisation de l'architecture CQRS est rendue possible grâce à plusieurs concepts tirés du Domain Driven Design (DDD).
\paragraph{Les aggrégats}
Il s'agit des objets sur lesquels agissent les commandes.
Ce concept, que l'on trouve dans le Domain Driven Design, correspond à un groupe d'objets que l'on peut traiter comme une unité.
Par exemple, l'aggregat 'Entreprise', que l'on retrouve dans l'espace recruteur de cadremploi.fr est constitué des objets 'NomEntreprise', 'AdresseFacturation' ou encore 'Logo'.
\paragraph{Les repositories (ou dépôts)}
Les différentes instances de cet aggrégat sont stockées dans un objet venu aussi du DDD, un repository (dépôt en français).
Ces objets servent d'intermédiaires entre \ref(subs:Le domaine){le domaine} et la base de donnée, qui est pour l'espace recruteur une base PostgreSQL.
On retrouve dans ces objets des méthodes du type 'create' permettant de créer une nouvelle instance d'un aggrégat ou 'save' qui permet de sauvegarder les changements relatifs à un aggrégat.
\paragraph{Récupération des données}
L'utilisation des repository via PostgreSQL est intéressante en écriture, mais ne permet pas une lecture intense et rapide.
L'équipe Cadremploi utilise Elasticsearch de manière à pouvoir indexer ces données et ainsi pouvoir chercher et trier rapidement les données dont elle a besoin.
C'est l'utilisation de cet outil qui permet d'exécuter des requêtes de manière extrêmement rapide.
%-------------------------------------------------------------------------
\subsubsection{Les commandes}
\label{subs:Les commandes}
L'utilisation du pattern Command est rendu possible grâce à plusieurs objets.
Un exécuteur de commande ("CommandExecutor") expose une méthode permettant d'exécuter une commande donnée.
Cet exécuteur se charge de
\begin{itemize}
  \item retrouver l'aggrégat visé par la commande s'il existe
  \item vérifier que l'auteur de l'action est autorisé à l'effectuer
  \item récupérer le CommandHandler associé à la commande, c'est à dire l'objet responsable de traiter la commande reçue
\end{itemize}


% parler des technos avant peut être (Akka, Angular, ...)
