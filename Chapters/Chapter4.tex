% Chapter 4

\chapter{Organisation et difficultés rencontrées} % Main chapter title

\label{difficultes} % For referencing the chapter elsewhere, use \ref{Chapter1}

\lhead{Cinquième partie. \emph{Organisation et difficultés rencontrées}} % This is for the header on each page - perhaps a shortened title

Mon stage de troisième année s'est dessiné autour de deux grands axes qui se sont entrecoupés tout au long des, au jour de ma soutenance, 25 semaines le composant.
La mise en place du nouvel espace recruteur restant l'axe le plus prenant, tandis que la création de la nouvelle Home de Cadremploi qui sera exposée au grand public le 15 octobre est arrivée plus tard et ne m'a, pour le moment, pas demandé trop de participation.
Il m'a fallu, quoi qu'il en soit, m'organiser et gérer mon temps correctement pour pouvoir mener à bien les tâches qui m'incombaient et surmonter les problèmes que je rencontrais.

%-------------------------------------------------------------------------------
%-------------------------------------------------------------------------------
\section{Arrivée et période d'adaptation}
%-------------------------------------------------------------------------------
\subsection{L'environnement}
Comme écrit au début de ce rapport de stage, j'ai dû me familiariser avec les outils spécifiques au développement d'application web Cadremploi dès le début de mon stage avec lesquels j'étais étranger.
La lecture de documents en ligne ainsi que les explications de mon maître de stage Laurent PAGEON et des autres membres de mon équipe m'ont permi de me familiariser bien plus facilement avec cet environnement et commencer à travailler de plus en plus de manière autonome.
%-------------------------------------------------------------------------------
\subsection{Notes}
J'ai dès le début de mon stage pris des notes sur un carnet, puis en ligne (sur l'application Evernote) sur ce que j'apprenais et que je faisais.
Cela m'a permis aussi de me rendre compte du temps que je prenais à effectuer une tâche et me forçait à réfléchir à une stratégie lorsque je me trouvais face à un problème.
Ce temps que je prenais à effectuer une amélioration sur l'Espace Recruteur était clairement plus long au début pour moi que pour les autres membres de l'équipe, mais avec du temps et de l'application, j'ai pu gagner en efficacité.
%-------------------------------------------------------------------------------
\subsection{Binômage}
C'est en binômant avec différents membres de l'équipe sur une même tâche que j'ai pu prendre connaissance de différentes parties de l'application.
Une des plus grosses difficultées que j'ai eu étant l'adaptation à l'architecture particulière de l'espace recruteur, l'aide des autres membres de mon équipe et ces binômages m'ont grandement aidés.
Ces différentes tâches effectuées à deux m'ont aussi, pendant les premiers mois, permis de découvrir les membres de l'équipe et leur force.
%-------------------------------------------------------------------------------
\subsection{Les stand-ups}
La communication au sein de l'équipe qu'amplifie la méthode Agile a aussi été bénéfique puisque le simple fait de m'exprimer sur les tâches dont je me chargeais avait pour effet de déclencher des remarques constructives de la part de mes collègues qui n'hésitaient de plus pas à prendre du temps pour m'expliquer des points complexes.
La migration de l'application de Play 2.3 à Play 2.4, source de nombreux changements, m'a posé de nombreux soucis et j'ai compris, un peu tard, que je n'avais les connaissances suffisantes pour m'en charger.
C'est plus tard durant mon stage que j'ai pu commencer à leur rendre la pareille.
