% Chapter 4

\chapter{Le stage dans ma formation} % Main chapter title

\label{formation} % For referencing the chapter elsewhere, use \ref{Chapter1}

\lhead{Quatrième partie. \emph{Le stage dans ma formation}} % This is for the header on each page - perhaps a shortened title

Durant ma formation, j’ai acquis beaucoup de connaissances théoriques et ai pu mettre pratique certaines d’entre elles lors de projets ou TP.
Répondre à la demande d’un client dans les délais impartis, communiquer ses idées et travailler en équipe avec des partenaires variés tels sont les objectifs du stage.


%----------------------------------------------------------------------------------------
%----------------------------------------------------------------------------------------
\section{Auto-évaluation}
%----------------------------------------------------------------------------------------
\subsection{Une nouvelle achitecture}
\label{sub:Une nouvelle achitecture}
%----------------------------------------------------------------------------------------
\paragraph{}
L'équipe Cadremploi fonctionne en suivant une méthode Agile, et les différentes évolutions effectuées sur le produit son répertoriés sur des tickets.
J'ai pendant un certain temps, travaillé en binôme avec différents membres de l'équipe (Laurent, Brice, Vincent, ...).
Cette période en début de stage m'a permi de découvrir l'application déjà complexe, surtout par le fait qu'elle utilisait une architecture que je n'avais jamais rencontré avant et nécessitait ainsi un vocabulaire tout nouveau pour moi.
C'est après cette période de presque un mois que j'ai commencé à prendre des tickets seul, comme un membre "normal" de l'équipe.
%----------------------------------------------------------------------------------------
\paragraph{}
J'ai ainsi réussi à progresser, de manière à prendre mes repères sur l'espace recruteur de Cadremploi.
Mes tâches me prenaient au début bien plus de temps que les autres membres de l'équipe, mais je parviens vers la fin de mon stage, avec une compréhension toujours incomplète, mais bien plus globale de l'application, à implémenter des solutions dans un temps correct.
%----------------------------------------------------------------------------------------
\subsection{Points problématiques}
\label{sub:Points problématiques}
%----------------------------------------------------------------------------------------
\paragraph{}
Je m'efforce à respecter les règles que se fixe l'équipe et rédige du code compréhensible.
Néanmoins, je manque même fin août, d'assez de rigueur quant aux tests que j'effectue sur la tâche sur laquelle je travaille.
De plus, mon manque de connaissance de l'application m'a parfois fait écrire du code ayant des conséquences sur des parties de l'application dont je ne soupçonnais pas l'existence.
Pour pallier à ces problèmes, je me suis occupé pendant une semaine de réécrire les tests unitaires, conformément aux normes que s'était fixé l'équipe.
De plus, je me suis documenté sur l'architecture qui était la notre de manière à mieux comprendre les mécanismes qui la composait.
%----------------------------------------------------------------------------------------
\paragraph{}
Bien que l'équipe Cadremploi soit très accessible, j'avais envie d'apprendre de manière la plus autonome possible sur l'application de manière à ne pas trop déranger.
Mon comportement s'est avéré réellement problématique notamment au moment où je me suis occupé de migrer l'application de la version 2.3 à 2.4 du framework Play.
Cette migration apportait beaucoup de changement, et demandait alors une connaissance de l'application et sur certaines technologies que je n'avais pas (Injection de dépendances, \ldots).
Je me suis rendu compte tard que j'avais besoin d'aide et je pense avoir fait perdre du temps à l'équipe bien que cette tâche ne soit pas bloquante.
C'est grâce à l'aide d'une personne qui en savait plus que moi que j'ai pu m'en sortir.
Lorsque moi aussi j'en ai su assez pour pouvoir partager mes connaissances, sur la compréhension de l'architecture, sur les structures Scala par exemple, j'en ai aussi fait profiter l'équipe en discutant sur des parties de l'application qui me semblait améliorables ou en binômant sur une tâche.
%----------------------------------------------------------------------------------------
\subsection{Evaluation}
\label{sub:Evaluation}
J'ai ainsi rencontré plusieurs difficultés lors de ce stage, et je n'ai pas su toutes les surmonter de la meilleure façon, mais il me semble en être sorti à chaque fois grandit.
J'ai commis des erreurs aussi, plusieurs, mais j'ai appris par elles, et grâce à mon équipe compréhensive



%----------------------------------------------------------------------------------------

\section{Résultats et prolongements possibles}
le site tourne et continuera de fonctionner encore

%----------------------------------------------------------------------------------------
%----------------------------------------------------------------------------------------

\section{Ce qui m'a été le plus utile dans ma formation}
J'ai pu voir lors de mon stage l'utilité des cours suivis à l'EISTI qui se sont montrés tant utiles qu'actuels.
En effet, je n'ai utilisé que très peu d'outils ou de technologies dont je n'avais jamais entendu parler
Scala, Akka, Veille technologique

%----------------------------------------------------------------------------------------

\section{Ce que m'a apporté ce stage pour le reste de ma carrière}
Git, veille technologique, architecture, scala
