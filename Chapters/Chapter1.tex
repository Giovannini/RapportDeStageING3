% Chapter 1

\chapter{Recherche du stage} % Main chapter title

\label{presentation} % For referencing the chapter elsewhere, use \ref{presentation}

\lhead{Première partie. \emph{Présentation de l'entreprise}} % This is for the header on each page - perhaps a shortened title

%----------------------------------------------------------------------------------------

\section{Introduction}

Le stage est une période très attendue de notre cursus au sein de l’EISTI car il nous permet de découvrir le monde du travail, notre futur métier et bien sûr d’appliquer les compétences que nous avons acqéries tout au long de notre scolarité.
De plus, le stage de dernière année se déroule sur une période de 20 à 22 semaines, une durée longue qui permet d'exploiter les avantages d’un stage et va nous permettre d’avoir un recul important à l’issue de celui-ci.

%----------------------------------------------------------------------------------------

\section{La façon dont j'ai procédé}
Le marché des offres de stage est bien plus caché que celui des offres d'emploi.
En effet, beaucoup d'offres ne sont pas publiées, ce qui rend la recherche bien compliqué.
Je vais expliquer dans la partie qui suit la façon dont je m'y suis pris pour trouver mon stage de fin d'étude.

\subsection{Mes motivations}
J'ai effectué l'an dernier un stage dans une très grosse entreprise, comprenant des équipes importantes et utilisant des méthodes de management anciennes.
J'ai beaucoup travaillé seul en parfaite autonomie mais sans réel contacts avec les autres équipes.
Je souhaitais ainsi cette année effectuer mon stage dans une structure plus petite, de la taille d'une startup ou d'une PME.
Toujours en opposition avec ce stage précédent, je souhaitais travailler en équipe, sur tes technologies nouvelles et en suivant des méthodes de management agiles.
Enfin, Scala est un langage qui m'a particulièrement intéressé cette année et l'utiliser pour les projets que nous avons effectués m'a motivé à chercher un stage dans lequel je pourrais m'y améliorer.

\subsection{Les entreprises contactées}
Le réseau de l'EISTI m'a proposé de nombreuses offres de stage différentes.
J'ai ainsi contacté des entreprises dont je recevais les offres par mail, qui étaient pour la plupart de grosses ESN.
Ces dernières sont connues pour employer assez facilement les eistiens; ainsi, bien que complètement éloignées des souhaits que j'ai énnoncé, ces entreprises m'assuraient.
J'ai aussi contacté quelques startup et PME, trouvées sur internet, en postulant à des offres de CDI ou en déposant des candidatures ouvertes.
J'ai contacté des entreprises implantées sur Pau, Toulouse et Bordeaux et quelques une sur Paris.

\subsection{Les résultats de mes recherches}
J'ai reçu assez rapidement des réponses de la part des ESN contactées et même effectué quelques entretiens avec certaines, mais les sujets que l'on me présentait ne correspondaient pas à mes critères du tout.
La plupart des projets se faisaient en J2E ou en C\#.
Les entreprises plus petites que j'ai contacté refusaient souvent mes demandes en tant que stagiaire puisqu'un CDI était recherché.
Une partie de ces entreprises m'ont recontactées pendant mon stage via le mail que je leur avait laissé ou bien via Linkedin.
FIGARO CLASSIFIEDS est l'une de ces PME que j'avais contacté pour un stage à la place d'un CDI.
L'entreprise m'a recontacté et nous avons organisé une rencontre via Skype très rapidement avec une responsable Ressources Humaines et monsieur Morel.
J'ai su rapidement, une semaine après l'envoi de mon premier mail, que j'étais pris pour ce stage.

%----------------------------------------------------------------------------------------

\section{Conclusion}

Grâce aux projets innovants réalisés, au stage de deuxième année et à mon expérience universitaire en Ecosse, j'ai pu présenter aux entreprises que je rencontrais un CV intéressant pour un étudiant en fin de cursus, puisque mes expériences n'étaient pas uniquement théoriques et qu'elles étaient enrichissantes.
Ainsi j'ai eu la possibilité de choisir un stage qui m'intéressait parmi plusieurs offres.
