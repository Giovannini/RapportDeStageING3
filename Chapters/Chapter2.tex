% Chapter 2

\chapter{L'environnement du stage} % Main chapter title

\label{environnement} % For referencing the chapter elsewhere, use \ref{Chapter1}

\lhead{Deuxième partie. \emph{L'environnement du stage}} % This is for the header on each page - perhaps a shortened title

%----------------------------------------------------------------------------------------

\section{Description objective de l'entreprise}

Petit tour rapide du Figaro

%----------------------------------------------------------------------------------------

\subsection{L'entreprise en général}

\subsubsection{Secteur d'activité}

concurrents, marché, évolution, ...
FIGARO CLASSIFIEDS, filiale du GROUPE FIGARO, est une des sociétés Internet les plus importantes en France, avec 60 M\texteuro de C.A., 350 collaborateurs et 3,5 millions de visiteurs uniques dédupliqués par mois sur l’ensemble de leurs sites.
Cette entreprise est présente sur 3 gros secteurs: l’Emploi, la Formation et l’Immobilier.
N°1 des "Quality Classifieds" en France, elle a pour ambition de proposer aux internautes/mobinautes et aux professionnels "le meilleur des médias et des solutions d’annonces classées".
Ses marques-phares (CADREMPLOI, KELJOB, LE FIGARO ETUDIANT, EXPLORIMMO, PROPRIETES DE FRANCE…) allient puissance, affinité CSP+ et influence, comme autant de facteurs de différentiation par rapport à nos concurrents.

FIGARO CLASSIFIEDS réalise 80\% de son chiffre d’affaires sur Internet, contribuant au développement numérique du GROUPE FIGARO, dont plus de 20\% du chiffre d’affaires total est réalisé sur Internet.

\subsubsection{Métiers}

On retrouve différents corps de métiers dans cette entreprise, puisque 5 grandes directions se côtoient:  Digital, Marketing, Communication et Édition, Ressources Humaines et Contrôle de Gestion.
Je faisais partie, en ce qui me concerne, du pôle Digital.

\subsubsection{Chiffre d'affaire}

60M€

\subsubsection{Effectifs}

PME de 350 collaborateurs

\subsubsection{Organisation interne}

FIGARO CLASSIFIEDS est dirigé par Thibaut GEMIGNANI et est organisé autour de trois gros secteurs que sont l'Emploi, la Formation et l'Immobilier. Autour de ces trois grands axes, on retrouve 5 grandes directions transverses:

\subsubsection{Relations du groupe}

blablabla Figaro, blablabla Serge Dassault

%----------------------------------------------------------------------------------------

\subsection{L'entreprise et son informatique}
L'informatique occupe une place très importante dans le département Digital de FIGARO CLASSIFIEDS.
Cette branche a majoritairement vocation a développer des applications web, c'est pourquoi de nombreux outils sont disponibles.

\subsubsection{Outils, technologies et méthodes}
\label{subs:Outils, technologies et methodes}
La plupart des employés a accès à un ordinateur, et un compte est attribué à l'arrivé, lui permettant de gérer notamment ses mails ou ses jours de congés.
Les développeurs sont pour la majorité sous Ubuntu et développent via l'IDE IntelliJ Idea pour lequel une license est disponible.
L'utilisation de l'ordinateur à disposition est plutôt libre, ce qui permet d'installer des logiciels sans avoir à passer par des Demandes de Prestation Informatique.
La majorité des équipes échelonne sa mise en production sous plusieurs étapes en déployant une nouveauté sur des serveurs internes particuliers avant de procéder au déploiement public.
Cela permet de sécuriser les nouvelles versions du logiciel puisqu'il est ainsi possible de s'apercevoir d'un problème avant qu'il ne soit visible par le public.
C'est via Jenkins, un outil d'intégration continu que les différentes release d'une application sont gérés.
Il s'agit d'un outil simplement mentionné en cours que je n'avais jamais utilisé auparavant.

\subsubsection{Outils de travail collaboratifs}
\label{subs:Outils de travail collaboratifs}
Des outils sont aussi disponibles pour permettre aux équipes de travailler ensemble plus facilement.
Redmine est un logiciel de gestion de projet qui est utilisé chez Figaro CLASSIFIEDS pour notamment lister les demandes client et qui permet de rendre compte en ligne du travail effectué.
Lorsque l'on s'occupe d'une tâche, qu'on la termine ou que l'on a besoin de plus de renseignement, on le mentionne sur Redmine, ce qui permet de centraliser l'information et de la rendre disponible au reste de l'équipe.
Git est aussi utilisé pour mettre en commun le code écrit par toute l'équipe.
Il s'agit d'un outil beaucoup utilisé à l'EISTI mais d'une façon très simpliste, et j'ai appris au cours de mon stage de nombreuses nouvelles possibilité pour cet outil.
Au cours de mon stage, l'équipe Cadremploi dans laquelle je travaille s'est aussi mise à utiliser Slack, une sorte d'application de messagerie assez informelle qui permet de centraliser les échanges de l'équipe en un seul endroit.
Il s'agit d'un outil que j'ai utilisé avec mon équipe au cours de mon PFE et qui s'est avéré très simple d'utilisation et très utile pour partager de l'information.


%----------------------------------------------------------------------------------------

\section{L'environnement de travail}
%----------------------------------------------------------------------------------------
\subsection{Les relations humaines au sein de l'entreprise}
Les membres des nombreuses équipes se côtoient sereinement, se respectent et s'entraident.

\subsection{Relations de la direction avec le reste du personnel}
Mails, ils viennent nous voir, ...

\subsection{Relations entre services, départements et divisions}
On se côtoit souvent pour échanger, souvent par le biais de Laëtitia, open space, entente cordiale, on va faire du sport ensemble

\subsection{Relations entre les différentes catégories de personnel et les différents niveaux hiérarchiques}
Partie qui semble inutile, déjà traité



%----------------------------------------------------------------------------------------

\section{Mon intégration dans l'entreprise}

Je dit bonjour à tout le monde et je tiens la porte.
