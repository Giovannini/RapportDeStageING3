% Chapter 5

\chapter{Le stage dans ma formation} % Main chapter title

\label{formation} % For referencing the chapter elsewhere, use \ref{Chapter1}

\lhead{Quatrième partie. \emph{Le stage dans ma formation}} % This is for the header on each page - perhaps a shortened title

Durant ma formation, j’ai acquis beaucoup de connaissances théoriques et ai pu mettre pratique certaines d’entre elles lors de projets ou TP.
Répondre à la demande d’un client dans les délais impartis, communiquer ses idées et travailler en équipe avec des partenaires variés tels sont les objectifs du stage.


%---------------------------------------------------------------
%---------------------------------------------------------------
\section{Ce qui dans ma formation m'a été le plus utile}

%---------------------------------------------------------------
\subsection{Programmation Scala et Framework Play}
\label{sub:Programmation Scala et Framework Play}
%---------------------------------------------------------------
\paragraph{}
La majorité du code que j'ai eu à écrire durant ce stage l'a été en Scala sous le framework Play, ainsi les deux coursen traitant m'ont été vraiment utiles.
Plus que de simplement me servir de ce que j'avais appris, ils ont été une base grâce à laquelle j'ai pu continuer à apprendre via des blogs ou des conférences en ligne.
Il m'a permis de progresser tant au niveau de la clarté de l'écriture du code qu'en connaissance sur les structures et les singularités que ce couple présente.
%---------------------------------------------------------------
\paragraph{}
De plus, il m'a sensibilisé à l'utilisation de structures fonctionnelles, plus lisibles, courtes et explicites que les structures habituelles que l'on peut rencontrer dans un langage orienté objet.
Ce point est primordial lorsque l'on travail au sein d'un projet avec d'autres personnes puisque ce qui est écrit est très succeptible d'être repris plus tard par d'autres personnes.

%---------------------------------------------------------------
\subsection{Angular}
\label{sub:Angular}
La partie Frontend de l'application Espace Recruteur demandait l'utilisation d'Angular et ce cours m'a permis, sans réels problèmes, de participer à son écriture, même si la majorité de mes apports au projet se faisaient en backend.
En effet, en plus de simplement cabler les parties client et serveur de l'application, et de me cantonner au traitement des tâches backend, j'ai su, en grande partie grâce à ce cours, apporter mon aide pour l'implémentation de la partie front.
J'ai en effet pu être rapidement à l'aise avec cette partie de l'application.

%---------------------------------------------------------------
\subsection{Enjeux et perspectives du Cloud Computing}
\label{sub:Enjeux et perspectives du Cloud Computing}
%---------------------------------------------------------------
\paragraph{}
Ce cours avait pour but d'inciter à la veille technologique via la présentation de contenus vidéos ou audios dont le sujet portait sur les nouvelles technologies.
Il m'a poussé à m'intéresser à différentes technologies, à faire des recherches sur les technologies que je ne maîtrisais pas encore (Kafka, ELasticSearch, ...), mais aussi sur celles que je commençais à maîtriser pour aller plus loin encore (Scala, Git, Akka, ...).
%---------------------------------------------------------------
\paragraph{}
C'est entre autre avec ce cours quelque peu spécial que j'ai commencé à prendre l'habitude de lire sur des nouveautés, des blogs de grands du web et ainsi voir leur avis sur de nombreux outils qui étaient utilisés aussi chez FIGARO CLASSIFIEDS.
Il a été une aide à la construction d'un avis sur plusieurs technologies et méthodes utilisés dans l'équipe.
%---------------------------------------------------------------
\paragraph{}
Enfin, ils ont été une sorte d'introduction aux architectures où de nombreuses technologies se croisent, puisqu'on s'éloigne des structures avec trois composants: frontend, backend et base de donnée pour quelque chose de plus souple mais aussi de plus efficace puisque des technologies dédiées sont utilisées pour des points particuliers, ce qui permet de gagner grandement en performance.

%---------------------------------------------------------------
\subsection{Autres}
\label{sub:Autres}
%---------------------------------------------------------------
\paragraph{}
De manière plus générale mais néanmoins importante, les différents projets, notamment de fin de 2e année et ceux de de 3e année (le PFE et l'application en partenariat avec CDiscount), ont été de belles occasions de travailler en équipe, de mettre en place des méthodes et d'expérimenter.
Ces projets m'ont ainsi beaucoup apporté en confiance quant à mes compétences mais aussi m'ont permi de m'exercer de façon concrète sur des technologies en dehors des cours parfois trop théoriques.
%---------------------------------------------------------------
\paragraph{}
Les cours de sécurité que j'ai suivi durant mon semestre de mobilité à Heriott Watt ainsi que durant ma dernière année à l'EISTI m'ont appris à repérer des failles courantes et à les éviter durant mon développement, ainsi qu'à en informer les autres.

%---------------------------------------------------------------
%---------------------------------------------------------------
\section{Ce qui dans mon stage me prépare à le suite de mon parcours}
%---------------------------------------------------------------
\subsection{Méthodes de travail}
\label{sub:Méthodes de travail}
%---------------------------------------------------------------
\paragraph{}
Ce stage était pour moi un premier vrai pas dans le monde du web en entreprise.
Il était donc une occasion de consolider mes bases en utilisant des technologies nouvelles.
Il a évidement été un test en soi de mes compétences, du niveau que j'ai atteint et de ma capacité à répondre à des requêtes, dans une structure et des enjeux plus grands que ce que j'avais pu expérimenter avec mes groupes de projets à l'EISTI.
C'est ainsi avec des exigences de respect de planning, des contraintes financières, dans un environnement de travail qui n'était pas celui que j'avais moi-même mis en place et avec des gens que je ne connaissais initialement pas que j'ai dû évoluer.
Ce fut donc une mesure concrète de la qualité de mes méthodes de travail et de leur mise en place dans un environnement nouveau autant qu'une prise de conscience concrète des besoins technologiques nombreux que demande la construction d'une application Web destiné à un public nombreux.
%---------------------------------------------------------------
\paragraph{}
Durant ce stage de fin d'étude, j'ai réalisé des tâches comme tout membre de l'équipe l'aurait fait et la prise de responsabilité a été bien plus grande que lors de mon stage de deuxième année.
La mise en place d'une nouveauté pouvait impacter dès la semaine suivante l'application en production, mais aussi directement mon équipe et il m'a été donc nécessaire de m'appliquer réellement, de mettre en place des tests, des vérifications importantes avant d'assurer la complétion d'une tâche.
Cette rigueur que je n'avais pas encore complètement a été un des grands apports de ce stage, en dehors des compétences technologiques et architecturales que j'ai pu développer.

%---------------------------------------------------------------
\subsection{Les acteurs de la création d'un produit}
\label{sub:Les acteurs de la création d'un produit}
Les réunions auxquelles j'ai assisté pendant mon stage, qu'il s'agisse de réunions d'information sur le plan marketing ou un rassemblement d'équipe pour débattre sur la stratégie à adopter pour une nouveauté, ont été très instructives.
Elles ont été une démonstration très intéressante des acteurs intervenant au cours de la mise en place d'un projet, d'un point de vue global, mais aussi des différents points de vue qu'il est possible d'adopter sur l'architecture d'une application comme l'est l'Espace Recruteur.
Interagir avec tous ces acteurs, les observer et travailler à leur côté a été une expérience réellement enrichissante.

%---------------------------------------------------------------
\section{Conclusion}
Ce stage m'a permis d'appliquer la théorie des cours au monde professionnel ainsi que de découvrir de nouvelles technologies et développer mes compétences.
L'Option Ingénierie du Coud Computing que j'ai suivie cette année a été un choix pertinent puisqu'il m'a mené dans une direction professionnelle qui correspond à mes objectifs: travailler dans un environnement novateur et dans lequel je puisse continuer d'apprendre et m'épanouir tant professionnellement que personnellement.
